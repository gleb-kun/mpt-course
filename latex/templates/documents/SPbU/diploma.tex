\documentclass[specialist,
substylefile = spbu.rtx,
               subf,href,colorlinks=true, 12pt]{disser}

%
% Основные пакеты
%

\usepackage[a4paper,
            mag=1000, includefoot,
            left=3cm, right=1.5cm, top=2cm, bottom=2cm, headsep=1cm, footskip=1cm]{geometry}
\usepackage[T2A]{fontenc}
\usepackage[utf8]{inputenc}
\usepackage[english,russian]{babel}

%
% Пользовательские пакеты
%

\usepackage{amssymb}
\usepackage{amsmath}

%
% Прочие настройки
%

\setcounter{tocdepth}{2} 	% Включать подсекции в оглавление

\pagestyle{plain}        	% Номера страниц снизу по центру с этой страницы

							% Возможности команды \pagestyle:
							% - 'plain' — отображает номер страницы снизу по центру (это по умолчанию для обычных страниц).
							% - 'headings' — номер страницы в верхнем углу, рядом с заголовками секций.
							% - 'myheadings' — пользовательские заголовки вверху (можно задавать вручную).
							% - 'empty' — отсутствие номера страницы и заголовков (не отображаются ни номера, ни заголовки).

\graphicspath{{fig/}}

\begin{document}

%
% Титульный лист на русском языке
%

% Название организации
\institution{
	Санкт-Петербургский государственный университет\\
}

% Факультет
\faculty{
	Математико-механический факультет
}

% Кафедра
\department{
	Кафедра вычислительной математики
}

% Работа
\title{
	Выпускная квалификационная работа
}

% Тема
\topic{
	Анализ воздействия резиновых уточек на изменение динамики жидкостных потоков в ванне
}

% Автор
\author{
	Фамилия
	Имя
	Отчество
}

% Информация об авторе, учебная программ и т. п.
\group{
	Уровень образования: бакалавриат\\
	Направление 01.03.02 <<Прикладная математика и информатика>>\\
	Основная образовательная программа СВ.5004.2018 <<Прикладная математика и информатика>>
}

% Научный руководитель
\sa       {И.\,И.~Иванов}
\sastatus {Профессор кафедры такой-то,\\
	д.\,ф.-м.\,н., профессор}

% Рецензент
\rev      {П.\,П.~Петров}
\revstatus{Наиглавнейший научный сотрудник, ООО~<<Рога и Копыта>>\\
	к.\,ф.-м.\,н., доцент}

% Город и год
\city{Санкт-Петербург}
\date{\number\year}

\maketitle

%
% Титульный лист на русском языке на Английском языке
%

% Название организации
\institution{
	Saint Petersburg State University
}

% Факультет
\faculty{
	Faculty of Mathematics and Mechanics
}

% Факультет
\department{
	Department of Computational Mathematics
}

%  Работа
\title{
	Graduation Project
}

% Тема работы
\topic{Dummy Diploma on Random Matter}

% Автор
\author{Pupkin Vasiliy Vasilyevich}

% Информация об авторе, учебная программ и т. п.
\group{
	Level of education: Bachelor's degree\\
	Specialization 01.03.02 <<Applied Mathematics and Computer Science>>\\
	Graduate program СВ.5004.2018 <<Applied Mathematics and Computer Science>>
}

%% Scientific Advisor
\sa       {I.\,I.~Ivanov}
\sastatus {Professor, Some department}
%
%% Reviewer
\rev      {P.\,P.~Petrov}
\revstatus{The most leading research associate, LLC~``Horns and Hoofs''}
%
%% City & Year
\city{Saint Petersburg}
\date{\number\year}

\maketitle[en]

\def\thispagestyle#1{}
\tableofcontents

\chapter*{Введение}\addcontentsline{toc}{chapter}{Введение}

\chapter{Название первой главы}

\section{Секция 1}

Содержимое секции...

\section{Секция 2}

Содержимое секции...

\chapter{Название второй главы}

\section{Секция 1}

Содержимое секции...

\section{Секция 2}

Содержимое секции...

\chapter*{Заключение}\addcontentsline{toc}{chapter}{Заключение}

\renewcommand{\bibname}{Список литературы}
\begin{thebibliography}{1}
	
	\bibitem{Lvovsky}
	Львовский С. М. Набор и верстка в системе LaTeX. М.: МЦНМО, 2021. 398 с.
	
\end{thebibliography}

\end{document}
